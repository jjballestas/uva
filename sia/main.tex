% PLANTILLA APA7
% Creado por: Isaac Palma Medina
% Última actualización: 25/07/2021
% @COPYLEFT

% Fuentes consultadas (todos los derechos reservados):
% Normas APA. (2019). Guía Normas APA. https://normas-apa.org/wp-content/uploads/Guia-Normas-APA-7ma-edicion.pdf
% Tecnológico de Costa Rica [Richmond]. (2020, 16 abril). LaTeX desde cero con Overleaf (1 de 3) [Vídeo]. YouTube. https://www.youtube.com/watch?v=kM1KvHVuaTY Weiss, D. (2021).
% Formatting documents in APA style (7th Edition) with the apa7 LATEX class. https://ctan.math.washington.edu/tex-archive/macros/latex/contrib/apa7/apa7.pdf @COPYLEFT

%+-+-+-+-++-+-+-+-+-+-+-+-+-++-+-+-+-+-+-+-+-+-+-+-+-+-+-+-+-+-++-+-+-+-+-+-+-+-+-+

% Preámbulo
\documentclass[stu, 11pt, letterpaper, donotrepeattitle, floatsintext, natbib]{apa7}
\usepackage[utf8]{inputenc}
\usepackage{comment}
\usepackage{marvosym}
\usepackage{graphicx}
\usepackage{float}
\usepackage[normalem]{ulem}
\usepackage[spanish]{babel}
\usepackage{datetime}
\usepackage{tcolorbox}
\usepackage{tikz}
\usepackage{hyperref}
\usepackage{ragged2e}
\usepackage{etoolbox}
\usepackage{caption}
\usepackage[spanish]{cleveref}
\usepackage{graphicx}
\usetikzlibrary{shapes.geometric, positioning}
\selectlanguage{spanish}
\useunder{\uline}{\ul}{}
\newcommand{\myparagraph}[1]{\paragraph{#1}\mbox{}\\}
\hypersetup{
    colorlinks=true,
    linkcolor=blue,
    filecolor=magenta,
    urlcolor=blue,
    citecolor=blue,
}

% Justificar texto SIN eliminar indentación
\RaggedRight
\justifying
\setlength{\parindent}{0.5in}

% Definir título y título corto para APA7
\title{Diseño del sistema de interacción avanzada SmartSound}
\shorttitle{Diseño del sistema de interacción avanzada SmartSound}
\author{John Ballestas, Nataly Rocha}
\affiliation{Universidad de Valladolid}

% Desactivar mayúsculas automáticas en el encabezado
\makeatletter
\let\MakeUppercase\relax
\patchcmd{\ps@titlepage}{\MakeUppercase}{}{}{}
\patchcmd{\ps@otherpage}{\MakeUppercase}{}{}{}
\makeatother

% Configurar captions de figuras centrados con espacio
\captionsetup[figure]{justification=centering, belowskip=4pt}

% Activar numeración de secciones y alinear a la izquierda
\setcounter{secnumdepth}{3}  % Numerar hasta subsubsection
\makeatletter
\renewcommand{\section}{\@startsection{section}{1}{\z@}%
    {-3.5ex \@plus -1ex \@minus -.2ex}%
    {2.3ex \@plus.2ex}%
    {\raggedright\normalfont\large\bfseries}}
\renewcommand{\subsection}{\@startsection{subsection}{2}{\z@}%
    {-3.25ex\@plus -1ex \@minus -.2ex}%
    {1.5ex \@plus .2ex}%
    {\raggedright\normalfont\normalsize\bfseries}}
\renewcommand{\subsubsection}{\@startsection{subsubsection}{3}{\z@}%
    {-3.25ex\@plus -1ex \@minus -.2ex}%
    {1.5ex \@plus .2ex}%
    {\raggedright\normalfont\normalsize\bfseries}}
\makeatother

\begin{document}

    % First Page - Centered
    \begin{titlepage}
        \centering
        \vspace*{1cm}

        \includegraphics[width=0.2\textwidth]{./assets/uva-logo}\\[3cm]

        {\Huge Diseño del sistema de interacción avanzada SmartSound}\\[1.5ex]
        {\LARGE Memoria del proceso}\\[3cm]

        {\Large John Ballestas}\\
        {\Large Nataly Rocha}\\[3cm]

        {\Large\bfseries Escuela de Ingeniería Informática, Universidad de Valladolid}\\[0.5cm]
        {\Large\bfseries Diseño de sistemas de interacción avanzada}\\[0.5cm]

        \newdate{hoy}{\the\day}{\the\month}{\the\year}
        \newcommand{\nextyear}{\the\numexpr\the\year+1\relax}
        {\Large\bfseries\the\year{} -- \nextyear}\\[0.5cm]

    \end{titlepage}


% Índices
    \pagenumbering{roman}
    % Contenido
    \renewcommand\contentsname{\large Índice}
    \tableofcontents
    \setcounter{tocdepth}{2}
    \newpage
    % Figuras
    \renewcommand{\listfigurename}{\large Índice de figuras}
    \listoffigures
    \newpage

% Cuerpo
    \pagenumbering{arabic}

    \vspace*{\fill}
    \section{Introducción}

    El presente trabajo expone el proceso de diseño del sistema de interacción avanzada SmartSound, desarrollado
    como proyecto final de la asignatura \textit{Diseño de Sistemas de Interacción Avanzada} del Máster en
    Ingeniería Informática.
    En él se detalla el conjunto de etapas seguidas durante el proceso de diseño, empleando la metodología del doble
    diamante como marco de referencia.

    Los músicos independientes y conferencistas enfrentan desafíos técnicos que afectan directamente la calidad de
    su actuación cuando hacen presentaciones en vivo, especialmente en espacios pequeños como bares y salas sin
    tratamiento acústico.
    La ausencia de personal especializado en sonido y la necesidad de autogestión del audio generan situaciones
    de estrés, pérdida de tiempo y malas experiencias tanto para los artistas como para el público.

    Este documento presenta el proceso de diseño de SmartSound, un sistema de interacción avanzada orientado a mejorar
    la gestión del sonido en presentaciones en vivo para grupos musicales bajo contrato que operan sin técnico
    de sonido fijo.
    A través de una investigación contextual y entrevistas en profundidad, se identificaron las necesidades
    reales de los usuarios, se descartaron perfiles no pertinentes y se definieron oportunidades de diseño
    centradas en la simplificación técnica, la retroalimentación visual y el cumplimiento normativo.

    El enfoque metodológico combina técnicas de observación directa, análisis temático y herramientas de diseño
    centrado en el usuario, como mapas de empatía, escenarios y definición de personas.
    El resultado es una propuesta de sistema que busca que los músicos autogestionen el sonido y mejoren
    la experiencia del público elevando el nivel de sus presentaciones.

    \vspace*{\fill}

    \newpage

    \vspace*{\fill}
    \section{Problema}

    \subsection{Problema inicialmente abordado}
    Usuarios que dependen de amplificación sonora para su trabajo como pequeños grupos musicales, dúos, solistas o
    conferencistas independientes que se presentan en bares, casas o salas se enfrentan a problemas de sonido antes
    y durante sus presentaciones, como feedback
    \footnote{Retroalimentación acústica no deseada que produce un sonido agudo o zumbido cuando el micrófono capta el
    sonido amplificado de los altavoces.}, niveles mal calibrados, saturación u otro debido a la falta de un
    técnico de sonido por razones de presupuesto, tiempo o falta de personal técnico.
    En la mayoría de los casos, deben llevar su propio equipo básico de amplificación y una consola para ajustar
    volúmenes y ecualización, haciéndose responsables de todo el montaje y control del audio.

    El procedimiento habitual consiste en realizar una prueba de sonido antes del evento, ajustando manualmente los
    niveles de cada instrumento en un espacio vacío.
    Sin embargo, al comenzar la presentación, el ambiente cambia
    drásticamente debido a la presencia del público: el sonido se modifica, aparecen problemas de feedback,
    variaciones de volumen y falta de claridad entre instrumentos.
    Al no escuchar lo mismo que percibe la audiencia, muchas veces desconocen estos problemas,
    que son notados principalmente por el público.

    La situación se complica porque, una vez iniciado el show, no pueden detenerse fácilmente a corregir el sonido;
    en algunos casos, uno de ellos debe dejar de tocar para hacer ajustes, lo cual afecta la presentación.
    Además, cada canción puede requerir configuraciones distintas. Por ejemplo, una balada suave frente a un tema de
    rock es muy diferente.
    Lo que impacta directamente en la calidad de la experiencia del público y la profesionalización de sus
    presentaciones.

    En concreto el problema consiste en la autogestión ineficiente de la calidad del sonido en presentaciones en vivo
    por parte de usuarios que depende amplificación sonara como músicos y conferencistas independientes, debido a la
    falta de un técnico de sonido especializado y a las complicaciones logísticas que implica atender
    simultáneamente la interpretación y la gestión del audio durante el evento.

    \vspace*{\fill}
    \newpage


    \subsection{Suposiciones}
    Para plantear el problema de manera inicial se han identificado las siguientes suposiciones:

    \begin{itemize}
        \item Los músicos como grupos, dúos, solistas y conferencistas son usuarios a los cuales les preocupa mucho
        la calidad sonora.
        \item Los usuarios se autogestionan el sonido durante sus presentaciones.
        \item Los usuarios tendrán al menos conocimientos básicos sobre el uso de equipos tecnológicos y de sonido.
        \item Uno de los principales miedos de los usuarios es enfrentar problemas como feedback inesperado, voz
        enterrada o poco clara y mezclas desbalanceadas en sus presentaciones.
        \item Para los usuarios es complicado controlar los cambios acústicos cuando entra el público a la locación.
        \item La logística de realizar correcciones durante el show es complicado y frustrante para los usuarios.
        \item Las bandas, dúos o solistas tendrán instrumentos de audio aceptable.
        \item Los músicos estarán dispuestos a ceder el control de la mezcla a un tercero o automatizarlo.
        \item Los espacios donde se realizan las presentaciones permiten la instalación y conexión rápida de equipos
        sin demasiadas trabas técnicas.
        \item Los bares, salones o lugares pequeños no contarán con un sonidista profesional ni con un sistema de sonido
        propio, lo que obliga a los músicos a manejar el audio por sí mismos.
        \item Los usuarios se presentan con regularidad en sitios distintos y es complicado para ellos realizar la
        autogestión del sonido para cada presentación.
    \end{itemize}


    \newpage
    \section{Fase de descubrimiento}
    \subsection{Plan de investigación}
    \subsubsection{Objetivos}

    \begin{itemize}
        \item Definir con claridad el grupo de usuarios principal del sistema propuesto, depurando perfiles que no
        se alineen con sus objetivos y necesidades.
        Para ello se pretende contrastar y verificar los perfiles existentes, eliminando suposiciones sobre sus
        miedos o limitaciones, así como comprender mejor los contextos y espacios en los que estos usuarios se
        desenvuelven.
        \item Analizar las metas, motivaciones y necesidades de los usuarios en el marco de los eventos en vivo,
        junto con sus niveles de competencia técnica y la complejidad que estarían dispuestos a aceptar en la
        adopción de un nuevo sistema.
        \item Revisar las prácticas actuales de los usuarios, entendiendo de qué forma resuelven hoy el problema que
        el sistema pretende abordar e identificando los productos de sonido y herramientas que utilizan
        habitualmente en sus eventos.
        \item Validar las expectativas de los usuarios respecto a la información que esperan recibir en el caso de
        que surjan problemas de sonido en su actuación, de modo que el diseño final responda directamente a sus
        requerimientos reales y no a suposiciones iniciales.
    \end{itemize}

    \subsubsection{Participantes más adecuados para adquirir conocimiento:}
    Los participantes que pueden aportar información más valiosa son aquellos para quienes la amplificación resulta
    esencial en su trabajo y que actualmente ofrecen presentaciones o charlas en vivo.
    Aquí algunos ejemplos:

    \begin{itemize}
        \item Bandas independientes
        \item Músicos bajo contrato para eventos sociales
        \item Solistas
        \item Dúos que se presenten en varios lugares
        \item Dueños de los bares en los que se hacen presentaciones
        \item Conferencistas independientes
    \end{itemize}

    \subsubsection{Técnicas de recogida de datos}
    \myparagraph{Investigación Contextual}
    Esta técnica permitirá observar directamente a los usuarios en su entorno natural, facilitando la eliminación de
    suposiciones sobre sus comportamientos, necesidades y dificultades reales \cite{invContext}.

    Además, proporcionará información valiosa sobre los espacios físicos, y las condiciones técnicas en las que se
    desarrollan sus actividades y cómo resuelven el problema actualmente.

    Las entrevistas en esta investigación también nos serán útiles para identificar qué aspectos valoran más en sus
    eventos en vivo eliminando suposiciones de calidad de sonido, facilidad de instalación, control técnico y otros.

    \myparagraph{Focus group}
    Una sesión de grupo ayudará a discutir sus experiencias frente a ciertos escenarios y qué problemas mencionan en
    común \cite{focusGroup}.

    Se podrá escuchar cómo los usuarios discuten sus frustraciones o miedos frente a nuevas tecnologías, lo cual
    permitirá eliminar suposiciones sobre supuestos miedos o problemas actuales.

    Esta técnica es muy útil al permitir la comparación de experiencias dentro del mismo perfil objetivo, revelando
    miedos de adopción tecnológica, uso de lenguaje común y expectativas.
    Sirve para priorizar necesidades antes de prototipar.\\[1cm]

    \subsubsection{Secuenciación}
    Iniciaremos con la investigación contextual especialmente para cubrir el objetivo de descartar grupos de
    usuarios que no se alineen con el problema propuesto.
    Mediante la observación y entrevistas se recopilarán transcripciones, notas de observación y registro
    fotográfico que permitirán concluir el grupo de usuarios más adepto a la investigación.

    Una vez determinado el grupo de usuarios procederemos con el focus group, tener un único grupo de usuario
    permitirá comprender y eliminar suposiciones de usuarios comunes recopilando información sobre sus metas,
    miedos, motivaciones y cómo se soluciona el problema actualmente.

    Los pasos a seguir para cada técnica serán:

    \myparagraph{Investigación Contextual}

    \begin{itemize}
        \item Visitar varios sitios de eventos en vivo en donde encontremos a usuarios actuando en su entorno natural.
        \item Ganar contexto mediante observación, conocer los tipos de espacios y coordinar entrevistas con los usuarios.
        \item Realizar entrevistas con los usuarios previo a su prueba de sonido y posterior a su actuación.
        \item Recabar información mediante preguntas del perfil de usuario, eliminar suposiciones e identificar para
        qué usuario sería más útil.
        \item Entrevistar a sus dueños antes y después del espectáculo para eliminar o verificar suposiciones sobre
        necesidad y cómo realmente les afectan los problemas planteados.
        \item Entrevistar a asistentes al evento sobre la calidad percibida y si es algo que afecte en su decisión
        de visitar este tipo de eventos.
    \end{itemize}

    \myparagraph{Focus group}
    \begin{itemize}
        \item Realizar un focus group el grupo de usuario que hemos identificado como los más potenciales mediante
        la investigación contextual.
        Para que compartan sus experiencias, ideas, miedos, problemas, soluciones para eliminar suposiciones y
        verificar necesidades.
        \item Identificar miedos y retos de adoptar nuevas tecnologías para solucionar sus problemas.
    \end{itemize}

    \subsection{Evento de investigación}
    Hemos seleccionado la investigación contextual por las ventajas que nos va a brindar y la facilidad para
    realizarla en el corto tiempo de investigación.

    Para realizar la recogida de datos se coordinó sesiones de observación y entrevistas con diferentes
    grupos musicales de la ciudad de Valladolid y también entrevistas telefónicas a grupos musicales y un
    conferencista fuera de la ciudad.

    Al coordinar la investigación únicamente se les indicó a los grupos de usuario que se trataba de un estudio
    que estamos realizando para un sistema de diseño interactivo y que nuestro propósito era tratar de conocerlos más.
    Lo que queríamos lograr era tratar de sesgar lo menos posible al usuario, para que tuviéramos respuestas mucho
    más certeras~\cite{preguntas}.

    El objetivo principal de las entrevistas era entender detalladamente las metas, retos y necesidades
    técnicas y
    emocionales de los músicos durante presentaciones en vivo, con un enfoque especial en la gestión del sonido.

    Se buscaba identificar responsabilidades, experiencias previas, grado de control y estrés, y cómo interactúan con
    el equipo y el público para descartar usuarios y recabar información útil para el diseño. Para esto se
    definieron fases de la entrevista como (1) contexto y objetivos, (2) retos y responsabilidades, (3) habilidades
    Técnicas, (4) automatización y (5) perspectivas futuras y cierre.

    Las entrevistas fueron grabadas para procesarlas posteriormente y en el caso de llamada telefónica se realizó la
    recopilación de datos en el guion de preguntas que preparamos para cumplir con los objetivos.

    Las transcripciones de las entrevistas se pueden encontrar en el apéndice~\ref{app:transcripciones}.

    Se describe a continuación el evento de investigación y recogida de datos para cada grupo de usuario:

    \subsubsection{Grupo de covers bajo contrato con 5 integrantes}

    \begin{figure}[ht]
        \centering
        \includegraphics[width=0.7\textwidth]{./assets/caimanes}
        \caption{Caimanes del Esgueva en el bar Zvmo.}
        \label{fig:caimanes}
    \end{figure}

    Para este grupo de usuarios se coordinó la investigación con el grupo ``Los Caimanes del Esgueva'' (Figura~\ref{fig:caimanes}),
    se realizó la coordinación de la entrevista mediante un mensaje privado en Instagram, en donde fueron muy
    amables de aceptarla.

    El lugar en donde se realizó la observación fue en el bar Zvmo en Valladolid, y tuvimos la oportunidad de
    observarlos tanto en su prueba de sonido como posteriormente en su presentación en vivo.

    Esta observación fue ideal porque pudimos observar todo el proceso del montaje en escena, cómo interactúan entre
    ellos, sus responsabilidades y evidenciar que el problema planteado estaba presente para su contexto.
    \myparagraph{Observaciones realizadas:}
    Es una banda de rock formada por cuatro integrantes, de aproximadamente 35 años, que gestionan sus propios
    instrumentos y consola, pero dependen del sistema de sonido del bar donde se presentan.
    Estos bares están adaptados para eventos musicales y cuentan con altavoces PA, lo que facilita en parte la
    logística para el grupo.

    A pesar de saber usar tecnología, los músicos no tienen formación técnica en sonido y realizan ajustes por
    prueba y error, lo que provoca interrupciones y toma mucho tiempo.
    Durante las pruebas de sonido y el concierto tuvieron que detenerse varias veces para modificar volúmenes y
    solucionar problemas auditivos que afectan la experiencia del público y pueden llevar a que algunos asistentes
    se retiren.

    Una observación clave de la entrevista fue la importancia de cumplir con la regulación europea que establece un
    límite máximo de decibelios de 100 dB. Este hallazgo es relevante para el diseño de futuras soluciones, ya que
    el control del volumen se vuelve esencial en estos espacios.


    \subsubsection{Grupo profesional de 4 integrantes}

    \begin{figure}[ht]
        \centering
        \includegraphics[width=0.7\textwidth]{./assets/canoneros}
        \caption{Cañoneros en fiestas del barrio Pilarica}
        \label{fig:canoneros}
    \end{figure}

    Para este grupo de usuarios se coordinó la investigación con el grupo ``Cañoneros'' (Figura~\ref{fig:canoneros}), se realizó la
    coordinación de
    la entrevista mediante un mensaje privado en Instagram, en donde fueron muy amables de aceptarla.

    El lugar en donde se realizó la observación fue en la tarima de las fiestas del barrio La Pilarica en
    Valladolid, y tuvimos la oportunidad de realizar una entrevista previo a su presentación y observarlos
    posteriormente en su presentación en vivo.

    \myparagraph{Observaciones realizadas:}
    Son un grupo de integrantes de 45+ años y la dinámica del grupo es más profesional y sobre todo con un técnico de
    sonido.
    El show estaba perfectamente previsto y ellos estaban muy tranquilos.

    Cuentan ya con mayor presupuesto y el escenario donde se presentaba ya era una plaza mucho más grande
    y con espacio abierto.
    Cuando hubo problemas de sonido un par de veces, enseguida el técnico procedió a ayudarlo y a corregir los
    problemas.
    Ellos no tuvieron que parar el show en ningún momento.

    \subsubsection{Grupo de música independiente autofinanciado de 3 integrantes}

    \begin{figure}[ht]
        \centering
        \includegraphics[width=0.7\textwidth]{./assets/arowana}
        \caption{Arowana en el bar Bicoca}
        \label{fig:arowana}
    \end{figure}

    Para este grupo de usuarios se coordinó la investigación con el grupo ``AROWANA'' (Figura~\ref{fig:arowana}), se realizó la
    coordinación de
    la entrevista mediante un mensaje privado en Instagram, en donde fueron muy amables de aceptarla.
    El lugar en donde se realizó la observación fue en el bar Bicoca en Valladolid, y tuvimos la oportunidad de
    realizar una entrevista previo a su presentación y observarlos posteriormente en su presentación en vivo.
    \myparagraph{Observaciones realizadas:}
    Son una banda independiente de tres integrantes, mayores de 45 años, que gestionan su propio sonido con equipo
    analógico y están en proceso de crecimiento.

    Aunque el espacio suele ser reducido y el sonido no siempre es óptimo, tanto el grupo como el público valoran
    más la comunidad y el apoyo mutuo que la calidad técnica, por lo que las pausas para ajustes se aceptan como
    parte de la experiencia.


    \subsubsection{Dueño del bar y dependientes}

    \begin{figure}[ht]
        \centering
        \includegraphics[width=0.7\textwidth]{./assets/duenos-bar}
        \caption{Bar Zvmo.}
        \label{fig:duenos}
    \end{figure}

    Decidimos también incluir en nuestras observaciones a los dueños de las locaciones en las que se presentan los
    grupos, debido a que teníamos la suposición de que ellos ven este tipo de espectáculos de manera continua y
    podríamos obtener mejor contexto.
    No se coordinaron entrevistas, únicamente se procedió a solicitar un tiempo breve de entrevista mientras
    realizamos la observación de los grupos.
    Se entrevistó al dueño del bar Radial y a la dependienta del bar Zvmo (Figura~\ref{fig:duenos}).
    \myparagraph{Observaciones realizadas:}
    Los resultados fueron menos concluyentes.
    Muchos dueños no se encuentran presentes durante las presentaciones y los empleados suelen rotar con frecuencia,
    lo que limita la obtención de información contextual profunda.

    Se percibe una preocupación general por mantener el volumen dentro de los límites legales y por el impacto del
    sonido en la experiencia del público, pero no un conocimiento técnico detallado.
    Se requiere más investigación específica para comprender mejor las necesidades de este perfil.


    \subsubsection{Dúo musical bajo contrato de covers y originales}
    Se coordinó la entrevista mediante Instagram y fue realizada a través de llamada telefónica.
    Diego A, el entrevistado, es uno de los integrantes de un dúo musical en Ecuador.
    Se han presentado en bares pequeños y medianos hasta en tarimas mucho más grandes en coliseos y plazas.
    Son un dúo de música independiente y llevan alrededor de 5 años de carrera.
    \myparagraph{Observaciones realizadas:}
    Aunque es un artista independiente y autogestionado con público que paga por sus shows, le preocupa
    especialmente la calidad del sonido y busca que sea impecable.
    Ha tenido malas experiencias, incluso con técnicos de sonido, por la falta de referencia de sus temas; por eso,
    prefiere automatizar tareas técnicas en lugar de delegarlas.

    \subsubsection{Conferencista independiente}
    Se coordinó la entrevista mediante Instagram y fue realizada a través de llamada telefónica.
    Fernanda A. es una conferencista Ecuatoriana ubicada en Madrid, España.
    Trata  temas tecnológicos especialmente en el ámbito del desarrollo de software, ha realizado ya más de 15
    conferencias a nivel internacional.
    \myparagraph{Observaciones realizadas:}
    No considera el sonido como un aspecto prioritario.
    Su principal preocupación es que el mensaje sea claro y la audiencia comprenda el contenido, delegando
    completamente los aspectos técnicos a los organizadores.
    Sus motivaciones y miedos están completamente alineados con el contenido de la charla, el sonido es algo que da
    por hecho en las conferencias.
    No tiene conocimientos básicos técnicos de sonido

    \subsection{Análisis de recogida de datos}
    Los datos recogidos de la investigación contextual fueron de tipo cualitativo por lo que elegimos la metodología
    de Análisis Temático (Thematic analysis) para su análisis.
    Se extrajeron las partes más relevantes de las transcripciones, clasificándolas posteriormente
    a través de códigos, agregando las temáticas más relevantes \cite{interactionD} y
    finalmente definiendo las conclusiones para la
    investigación de usuarios \cite{thematic}.

    \subsubsection{Tablero del análisis temático}
    En el siguiente enlace se puede visualizar el tablero con todo el trabajo realizado para el análisis de datos
    \href{https://miro.com/app/board/uXjVJAPP6xA=/?moveToWidget=3458764645785384622\&cot=14}{[Tablero Miro]} y de
    igual manera en (Figura~\ref{fig:tematico}) se puede visualizar los temas y códigos recopilados.

    \begin{figure}[ht]
        \centering
        \includegraphics[width=0.8\textwidth]{./assets/analisisTematico}
        \caption{Análisis temático - temas y códigos}
        \label{fig:tematico}
    \end{figure}

    \subsubsection{Conclusión del análisis de datos}

    \begin{enumerate}
        \item La investigación permitió contrastar las suposiciones iniciales sobre los perfiles de usuario y profundizar en
        la comprensión de sus metas, necesidades y contextos de uso.
        Las entrevistas y observaciones redefinieron las oportunidades de diseño del sistema, aportando una visión más
        precisa sobre qué grupos son realmente relevantes.
        \item En primer lugar, se confirmó que los conferencistas no priorizan el sonido en su trabajo, lo que lleva a
        descartarlos como usuarios pertinentes.
        En el caso de los músicos independientes, emergió una dimensión social
        en la que prioriza el apoyo a la música antes que lo técnico.
        Por ello, se decidió excluir a este grupo de usuarios, así como a los conjuntos profesionales con técnicos de
        audio dedicados, ya que cuentan con recursos y conocimientos avanzados.
        \item Se puede definir que el grupo de usuarios más adecuado para continuar el desarrollo
        del sistema son los grupos bajo contrato de 3 a 5 integrantes que realizan presentaciones en bares o venues
        medianos, sin contar con técnico de sonido fijo.
        Tiene una muy buena motivación por tener un buen momento en su presentación, brindar un show de calidad al
        público y cuentan con un sentido de profesionalismo y responsabilidad sobre el aspecto técnico de su show.
        \item Los resultados validan que la mayoría de los músicos posee solo conocimientos técnicos básicos.
        Sin embargo, esta limitación se traduce en dificultades concretas al manejar equipos profesionales, ocasionando
        errores de configuración y pérdidas de tiempo en las pruebas de sonido.
        Este hallazgo refuerza la necesidad de un sistema que simplifique el proceso de calibración y brinde
        retroalimentación guiada, reduciendo la dependencia de la experiencia técnica.
        \item Respecto a las bandas de covers y grupos bajo contrato, se comprobó que el sonido es un factor crítico para su
        desempeño y reputación.
        A diferencia de las bandas pequeñas o independientes, estas dependen directamente de la calidad sonora para
        sostener su profesionalismo y obtener futuras contrataciones.
        El número de instrumentos no resulta determinante; lo que realmente diferencia a los grupos es su nivel de
        profesionalización y autogestión.
        \item Por otra parte, los datos sobre dueños y empleados de bares fueron menos concluyentes, debido a la limitada
        muestra y a la alta rotación del personal.
        Aun así, se observó que en locales pequeños suele no haber sonidista, por lo que los músicos asumen la gestión
        del audio.
        Este patrón confirma la relevancia del sistema en entornos con recursos técnicos limitados.
        \item Finalmente, se identificó una preocupación compartida por músicos y dueños de locales respecto al cumplimiento
        de los límites de decibelios.
        Esta evidencia abre una oportunidad de diseño para incorporar funciones de monitoreo y control automático del
        volumen dentro del sistema.
        \item
    \end{enumerate}

    \subsection{Definición del espacio del problema}
    \subsubsection{Persona}

    Acorde al análisis de datos hemos definido a la persona Eddie (Figura~\ref{fig:persona}) para continuar con el diseño del
    sistema propuesto.
    En este \href{https://miro.com/app/board/uXjVJAPP6xA=/?moveToWidget=3458764644105155139\&cot=14}{[Tablero Miro]} se
    puede navegar por la ficha de la persona a profundidad.

    \begin{figure}[H]
        \centering
        \includegraphics[width=0.9\textwidth]{./assets/persona}
        \caption{Eddie - Persona definida}
        \label{fig:persona}
    \end{figure}

    \subsubsection{Mapa de empatía}

    El mapa de empatía de Eddie (Figura~\ref{fig:empatia}) permitió comprender a fondo cómo piensa, siente y actúa durante sus
    presentaciones en
    vivo.

    \begin{figure}[H]
        \centering
        \includegraphics[width=1\textwidth]{./assets/empatia}
        \caption{Mapa de empatía}
        \label{fig:empatia}
    \end{figure}


    Revela que es un músico apasionado y comprometido, cuyo principal objetivo es conectar con el público y ofrecer
    un show impecable, aunque su conocimiento técnico limitado les genera ansiedad y pérdida de tiempo frente a
    problemas de sonido.
    Valoran la calidad más que el volumen, buscan estabilidad y retroalimentación visual debido
    al uso de in-ears, y se apoyan en la autogestión mediante prueba y error.
    En este \href{https://miro.com/app/board/uXjVJAPP6xA=/?moveToWidget=3458764644107141582\&cot=14}{[Tablero Miro]} se
    puede navegar por el mapa a profundidad.
    Este ejercicio, basado en datos reales, ayuda al equipo de diseño a empatizar con los usuarios, alinear una
    visión común, evitar sesgos y detectar áreas críticas de oportunidad, guiando el diseño hacia soluciones que
    simplifiquen la experiencia técnica y potencien la conexión emocional del grupo con su público.

    \subsubsection{Escenarios}

    \textbf{Escenario 1: Preparación antes del show (Presente)}

    Eddie llega con su grupo a un bar mediano donde se presentará esa noche.
    Eddie, guitarrista, comienza inspeccionando el lugar para entender la disposición del escenario y la posición del
    público.
    Luego instalan sus instrumentos, realizan las conexiones, encienden la consola y el sistema de amplificación (PA).

    El grupo inicia la prueba tocando una canción que conocen bien.
    Mientras suenan los primeros acordes, Eddie recorre el bar para escuchar desde distintos puntos y detectar
    posibles ajustes.
    De repente, aparece un feedback en el sistema y nadie sabe con certeza cuál es su causa.
    La banda se detiene y Eddie vuelve a la consola para revisar los niveles y conexiones.

    Cuando retoman, notan que el bajo no se escucha correctamente, por lo que deben detenerse otra vez para
    identificar el problema.
    Tras varios intentos y correcciones, el grupo vuelve a tocar, mientras Eddie recorre la sala observando si el
    sonido mejora.
    Finalmente, logran una mezcla aceptable y deciden dar por terminada la prueba.


    \textbf{Escenario 2: Durante el show monitoreo y resolución de problemas (Presente)}

    Eddie y su grupo ya están sobre el escenario.
    El bar está lleno y el público disfruta de las primeras canciones.
    Todo parece marchar bien hasta que, a mitad del show, el sonido comienza a distorsionarse y se escucha un fuerte
    feedback.
    Desde el escenario, los músicos notan que algo anda mal, pero no logran identificar la causa exacta.

    Eddie, se acerca rápidamente a la consola mientras el resto del grupo deja de tocar.
    En medio del ruido y la confusión, intenta ajustar los niveles y revisar las conexiones, pero el problema
    persiste unos minutos más.
    Durante esa pausa, parte del público se incomoda por la interrupción y algunos asistentes deciden abandonar el bar.

    Después de varios intentos, Eddie logra estabilizar el sonido y eliminar el feedback.
    El grupo retoma la presentación y aunque el show continúa sin más incidentes, la banda queda frustrada: el
    momento de tensión rompió el ritmo de la actuación y afectó la experiencia tanto para ellos como para la audiencia.

    \subsubsection{Mapa de recorrido}
    El mapa de recorrido a continuación (Figura~\ref{fig:recorridoPresente}) describe el comportamiento del usuario con base en su situación
    actual.
    En este \href{https://miro.com/app/board/uXjVJAPP6xA=/?moveToWidget=3458764645681572626\&cot=14}{[Tablero Miro]} se
    puede navegar por el mapa de recorrido de usuario a profundidad.

    El escenario explorado en este mapa se basa en el escenario 1 descrito anteriormente, en el cual podemos notar
    los diferentes problemas por los que pasa el usuario y sus oportunidades.

    \begin{figure}[H]
        \centering
        \includegraphics[width=1\textwidth]{./assets/recorrido-presente}
        \caption{Mapa de recorrido - Presente}
        \label{fig:recorridoPresente}
    \end{figure}

    \subsubsection{Requisitos de usabilidad}

    Tras la definición del problema, se obtuvo una comprensión más clara del usuario y de las dificultades que
    enfrenta durante la instalación y prueba de sonido.
    Se identificó que los usuarios experimentan incertidumbre al iniciar el proceso, especialmente al no contar con
    una guía clara o con los conocimientos técnicos necesarios, lo que genera inseguridad sobre la calidad del
    resultado final.
    Esta situación evidencia la necesidad de herramientas que orienten y simplifiquen las tareas, brindando
    confianza en cada etapa.

    A partir de este análisis, los requisitos de usabilidad definidos son:

    \begin{itemize}
        \item Los usuarios deben ser capaces de realizar la calibración inicial en menos de 5 minutos.
        \item Los usuarios deben ser capaces de sincronizar los implementos de sonido y de ayuda digital en menos de
        3 minutos
        \item El 90\% de los usuarios debe ser capaz de interpretar fácil y adecuadamente la información que le
        brinde el sistema de sonido
        \item Los indicadores visuales deben refrescarse al menos cada 3 segundos para mantener una percepción de
        tiempo real.
        \item El 80\% de las sesiones debe completarse sin interrupciones por fallos técnicos o de niveles.
        \item La aplicación debe ser operable con una mano y requerir como máximo tres interacciones para acceder a
        cualquier función clave (inicio de calibración, selección de preset, monitoreo).
    \end{itemize}

    \newpage
    \section{Fase de ideación y evaluación}
    \subsection{Soluciones planteadas}
    \subsubsection{Solución 1}
    \textbf{Smart Sound como Mezclador Digital Inteligente con Presets Adaptativos}
    Un técnico de sonido digital equipado con algoritmos de calibración automática y bancos de presets
    personalizados por canción puede simplificar la configuración inicial.
    Mediante reconocimiento de pistas referenciales y pre-ajustes, el sistema reacciona a la interacción del grupo y
    sus instrumentos, calibra inteligentemente los volúmenes y la ecualización de cada instrumento antes del show.
    Adapta los parámetros al ambiente en tiempo real utilizando sensores de decibelios y respuesta acústica.
    Siempre brindando una interfaz de retroalimentación a los usuarios, en donde pueden visualizar el estado del
    sonido desde la perspectiva del público y pueden saber todo el tiempo si todo va bien o algo está fallando.

    \paragraph{Escenario 1 (durante prueba de sonido) para esta solución:}

    Eddie llega con su banda a un bar mediano donde se presentarán esa noche.
    Tras instalar sus instrumentos, conectan la consola y los receptores de audio.
    Desde la tablet abren la aplicación SmartSound para iniciar la calibración del sonido.

    El sistema analiza automáticamente la acústica del lugar y muestra indicadores visuales que sugieren pequeños
    ajustes.
    Tras la autorización del usuario, SmartSound realiza las correcciones necesarias y confirma que la mezcla está
    equilibrada y dentro de los niveles de decibelios permitidos.
    La banda realiza una breve prueba de sonido sin interrupciones ni incertidumbre sobre la calidad final.

    \subsubsection{Solución 2}
    \textbf{Smart Sound como Sistema con Control mediante Wearables y Reconocimiento de Gestos}

    Permitir que los músicos administren la mezcla desde smartphones, tablets o wearables con controles gestuales
    sencillos (deslizar para subir/bajar volumen, toque para activar preset) elimina el temor a interfaces complejas.
    Integrando un asistente de voz, el sistema puede recibir comandos básicos como ``Claridad de voz'', ``Más volumen al
    teclado'' ayudando a los músicos a gestionar sus necesidades de sonido.

    \paragraph{Escenario (durante prueba de sonido) para esta solución:}

    Eddie y su banda llegan a un bar mediano para su presentación.
    Tras conectar los instrumentos, sincronizan sus dispositivos con el sistema SmartSound.
    Eddie usa su smartwatch, la vocalista una tablet y el guitarrista su smartphone.

    Durante la prueba de sonido, realizan ajustes sencillos con gestos.
    Eddie desliza el dedo para subir su guitarra, con comandos de voz la vocalista dice ``más claridad en la voz''.
    El sistema responde al instante, mostrando una confirmación visual y adaptando la mezcla automáticamente.

    \subsection{Solución seleccionada}

    La solución elegida es el sistema SmartSound, un asistente inteligente de calibración de audio diseñado para
    garantizar mezclas equilibradas y consistentes.
    Su función principal es ofrecer seguridad, autonomía y estabilidad sonora a músicos sin técnico de sonido.

    \paragraph{Mapa de recorrido de usuario}
    Este mapa de recorrido (Figura~\ref{fig:recorridoFuturo}) describe el comportamiento deseado de un usuario ante
    el sistema basado en esta solución.
    En este \href{https://miro.com/app/board/uXjVJAPP6xA=/?moveToWidget=3458764645640663142\&cot=14}{[Tablero Miro]} se
    puede navegar por el mapa de recorrido de usuario futuro a profundidad.

    \begin{figure}[H]
        \centering
        \includegraphics[width=1\textwidth]{./assets/recorrido-futuro}
        \caption{Mapa de recorrido - Futuro}
        \label{fig:recorridoFuturo}
    \end{figure}

    \subsubsection{Componentes principales del sistema}

    \begin{itemize}
        \item Receptores de audio distribuidos en la sala que capturan la mezcla desde la perspectiva del público.
        \item Consola central digital, donde se procesan los datos acústicos y se conectan los instrumentos.
        \item Aplicación móvil/tablet, que permite a los músicos controlar el sistema, recibir retroalimentación
        visual, cargar presets y guardar calibraciones personalizadas.
    \end{itemize}

    Se representan los elementos mencionados en el mapa conceptual (Figura~\ref{fig:concepto})

    \begin{figure}[H]
        \centering
        \includegraphics[width=1\textwidth]{./assets/mapaConcepto}
        \caption{Mapa conceptual de SmartSound}
        \label{fig:concepto}
    \end{figure}

    El sistema puede conectarse mediante red inalámbrica o cableada a la consola principal, integrándose con el
    equipo existente y ofreciendo una experiencia ``plug \& play'' para pequeños escenarios.

    \subsection{Estrategia de evaluación}
    Se realizará una \textbf{evaluación controlada} \cite{interactionD} en etapas
    tempranas del diseño, centrada en \textbf{identificar problemas de
    usabilidad} y evaluar la comprensión e interacción de los usuarios con la aplicación SmartSound

    \subsubsection{Objetivo principal}
    Detectar dificultades en la interacción con la aplicación durante el proceso de calibración y comprobar si los
    usuarios entienden la retroalimentación visual del sistema.

    \subsubsection{Objetivos específicos}

    \begin{itemize}
        \item Evaluar la facilidad para iniciar y completar el proceso de calibración.
        \item Analizar la claridad y utilidad de los indicadores e instrucciones del sistema.
        \item Evaluar la percepción de confianza del usuario en la calibración automática.
        \item Identificar mejoras en la interfaz y flujo de tareas.
    \end{itemize}

    \subsubsection{Restricciones}
    Limitaciones de tiempo, presupuesto y disponibilidad de usuarios reales impiden realizar pruebas de campo con
    bandas activas.
    Por ello, se utilizarán participantes representativos en entornos controlados (salas de ensayo).


    \subsection{Plan de evaluación}
    \subsubsection{Participantes}
    \myparagraph{Perfil:} músicos amateurs que ensayan regularmente en salas pequeñas.
    \myparagraph{Número:} entre 3 y 5 participantes individuales o grupos pequeños.
    \myparagraph{Selección:} por conveniencia (usuarios representativos pero accesibles).

    \subsubsection{Producto a evaluar}
    Prototipo ``Mago de Oz'' de SmartSound que simulará el comportamiento del sistema.
    La aplicación será creada con un mock interactivo de la app en Figma (modo prototipo).
    Utilizaremos consola digital real existente, receptores de audio simulados en cartón o impresión 3D e
    instrumentos y amplificación reales de la sala.

    \subsubsection{Tareas a realizar}

    Seguiremos el mapa de recorrido del usuario para prueba y calibración de sonido en la locación, en donde seguirán los siguientes pasos:

    \begin{itemize}
        \item Inspección del lugar
        \item Instalación de los instrumentos, conexiones y colocación de los receptores
        \item Abrir la aplicación Mago de Oz en la tablet y empezar a interactuar con ella
        \item Realizar la prueba de sonido
        \item Realizaremos un ajuste del sonido de manera manual como si fuera el ajuste automático de la consola, para brindar la sensación al usuario de su funcionamiento.
        \item Mostrar los resultados de la calibración en la aplicación
        \item Verificación del sonido
        \item Guardar la calibración realizada en la aplicación
    \end{itemize}

    El observador registrará tiempos, errores, expresiones faciales y comentarios (``pensar en voz alta'').

    \subsubsection{Aspectos concretos a consultar con los usuarios}

    Específicamente nos vamos a centrar en la usabilidad de la aplicación para esta evaluación, refiriéndose a que
    tan complicado es para los usuarios:

    \begin{itemize}
        \item Iniciar la calibración
        \item Comprender el proceso que la aplicación está realizando
        \item Que tan útil es la información que el sistema les está brindando
        \item Cómo perciben ellos que realmente una auto calibración está sucediendo
        \item Qué tan fácil fue verificar que el sonido realmente se calibró
        \item Cómo fue el proceso de guardar sus configuraciones
    \end{itemize}

    \subsubsection{Lugar de evaluación}
    La prueba se realizará en una sala de ensayo controlada, con condiciones acústicas realistas pero manejables.
    Idealmente, los participantes no deberían haber ensayado antes en ese lugar para reducir sesgos.

    \subsubsection{Datos a recoger}
    Es necesario recoger métricas basadas en los problemas que tuvieron los usuarios para realizar estas tareas y de
    observaciones realizadas en el proceso de evaluación.
    Creando estadísticas de los tipos de problemas detectados, sus tipos y su grado de severidad.
    Tras cada sesión, se aplicará una breve entrevista post-prueba para recoger percepciones, sugerencias y
    expectativas.

    \subsubsection{Conclusiones esperadas de la evaluación}
    Permitirá verificar si la interfaz transmite confianza, si la interacción resulta intuitiva y si
    el concepto de calibración automática se percibe como comprensible y útil.
    Los hallazgos orientarán mejoras antes de avanzar hacia un prototipo funcional.


    \newpage
    \section{Aspectos no tratados, cuestiones a seguir investigando}

    Si bien la investigación y el diseño desarrollados han permitido delimitar con precisión el grupo de usuarios y
    las principales oportunidades de mejora en la experiencia con el problema, existen aún líneas de exploración que
    no se han abordado en esta fase inicial y que resultan relevantes para consolidar el sistema SmartSound.

    En primer lugar, no se ha planificado aún una evaluación de campo real con grupos musicales en condiciones de
    actuación en vivo.
    La validación en escenarios controlados permitiría obtener información sobre la usabilidad de la interfaz, pero no
    sobre su comportamiento frente a variables ambientales reales, como ruido del público, reverberación o
    interferencias de señal.
    Este aspecto será clave para contrastar la eficacia del sistema y su grado de autonomía ante contextos
    acústicos y humanos cambiantes.

    De igual forma, no se ha profundizado en el impacto de la integración tecnológica con los diferentes modelos
    de consolas y equipos de audio ya existentes en el mercado.
    Es necesario continuar investigando la compatibilidad técnica y también hacer un mejor análisis de los competidores ya existentes.

    Otra línea pendiente es el estudio de la percepción emocional y cognitiva de los usuarios frente al
    uso de sistemas automáticos.
    Aunque los resultados muestran una actitud favorable hacia la automatización, sería valioso analizar cómo esta
    afecta la sensación de control y confianza del músico durante la actuación, así como el equilibrio entre
    autonomía tecnológica y creatividad artística.

    Por último se podría investigar en el futuro la adaptabilidad del sistema a otros perfiles de usuario, como
    técnicos de sonido noveles o productores audiovisuales, para explorar su aplicabilidad en entornos más amplios.

    \newpage
    \section{Conclusiones}
    El proyecto SmartSound siguió los principios del diseño de sistemas interactivos basados en el modelo del doble
    diamante \cite{diamante}, alternando fases de divergencia y convergencia para garantizar un proceso centrado en
    las personas y orientado a crear una experiencia de uso coherente, útil y significativa.

    En la fase de descubrimiento, se aplicaron técnicas cualitativas como investigación contextual que permitieron
    comprender el contexto real de los músicos y profesionales que gestionan su propio sonido.
    Este enfoque facilitó verificar suposiciones y profundizar en los factores emocionales, técnicos y sociales que
    condicionan la interacción con los sistemas de audio, cumpliendo con el principio de ``poner a las personas primero''.

    La fase de definición se basó en la síntesis de la investigación mediante personas, mapas de empatía y escenarios,
    que permitieron representar de forma empática y estructurada las metas y comportamientos del usuario principal,
    favoreciendo una visión compartida del problema y evitando soluciones descontextualizadas.

    Durante la fase de ideación, se generaron y evaluaron distintas alternativas aplicando los principios del Design
    Thinking \cite{interactionD}, priorizando la simplicidad, la adaptabilidad y la autonomía del usuario.
    Este proceso condujo a la selección de una solución coherente con los hallazgos: SmartSound como asistente
    inteligente de calibración y monitoreo de audio.

    Finalmente, la fase de evaluación, planteada con un prototipo tipo Mago de Oz, materializa la naturaleza
    iterativa del doble diamante, concibiendo el diseño como un proceso en constante validación.
    Las métricas de usabilidad, confianza y comprensión de la retroalimentación permitirán refinar la interfaz
    y mejorar la experiencia global.

    Muy lejos de ser un diseño terminado el proyecto ha seguido el proceso de diseño de los métodos centrados en las
    personas en la ingeniería de sistemas interactivos reforzando la conexión emocional entre el músico,
    su sonido y su audiencia.

    \newpage
% Referencias
    \renewcommand\refname{\textbf{Referencias}}

    \bibliographystyle{acm}
    \bibliography{refs}

    \newpage
    \appendix

    \section*{Apéndices}
    \addcontentsline{toc}{section}{Apéndices}

    % Aquí puedes agregar el contenido de los apéndices
    % Por ejemplo:
    % \subsection{Transcripciones de entrevistas}
    % \subsection{Guías de entrevista}

\end{document}
